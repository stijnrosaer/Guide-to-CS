\documentclass[a4paper]{article}

\usepackage[dutch]{babel}
\usepackage{graphicx}
\usepackage{color}

\begin{document}
	\title{The Hitchhiker's Guide To Computer Science}
	\author{
		Robbe Claessens - robbe.claessens@student.uantwerpen.be\\
		Joey De Pauw - joey.depauw@student.uantwerpen.be\\
		Stan Draulans - stan.draulans@student.uantwerpen.be\\
		Tim Leys - tim.leys@student.uantwerpen.be\\
		Stijn Vissers - stijn.vissers@student.uantwerpen.be
	}
	\date{}
	\maketitle
	
	\section{Intro}
		Studenten die hun opleiding secundair afronden en kiezen voor de opleiding informatica, komen terecht in een nieuwe omgeving. Zij moeten zich aanpassen aan een andere manier van lesgeven. Ze krijgen meer vrijheid en bijgevolg ook meer verantwoordelijkheid. Daarenboven moeten ze leren om zelfstandig projecten te maken. Het merendeel van de vakken is verbonden aan een praktische proef, een individuele programmeeropdracht of groepswerk. \\
		Om De eerstejaars te helpen aan het wenningsproces willen wij in het begin van het jaar enkele praktische "inleidingslessen" geven. Deze lessen hebben volgende doelen:
		\begin{itemize}
			\item Informatie geven over beschikbare programma's die het leven van de informaticus gemakkelijker maken. (bv.: git)
			\item De studenten wegwijs maken met services aangeboden door de universiteit.
			\item Ondersteuning bieden bij het werken met Linux.
			\item Het opbouwen van een band tussen de oudere studenten en de eerstejaars studenten. De mentor informatica van WINAK en de andere begeleiders maken zich bekend als aanspreekpunt voor vragen en suggesties.  
		\end{itemize} 
		Wij hebben de verzameling van deze sessies "The Hitchhiker's Guide To Computer Science" gedoopt.
		
	\section{Onderwerpen \& Planning}
		Het idee is om een wekelijkse les in te plannen van $\pm$1u vanaf de eerste lesweek. Bij voorkeur staat deze sessie als laatste les van de dag. Zo is er geen leegte in het lessenrooster voor degene die niet ge\"interesseerd zijn. De sessies lopen normaal ook niet tot het einde van het semester. \\
		In dit hoofdstuk staan de sessies beschreven die we zeker willen geven. De meeste staan los van elkaar. Een student kan dus perfect enkel naar de sessies komen waar hij/zij interesse in heeft. Bij elke sessie staan minstens 2 personen vermeld die hier verantwoordelijk voor zijn. Concreet houdt dit in dat zij het nodige materiaal voorzien, alles voorbereiden en de sessie leiden. Dit alles kan natuurlijk in samenwerking met de rest. De sessies staan in chronologische volgorde vermeld. Het is ook mogelijk om een extra sessie te doen over een onderwerp gekozen door de studenten.
		
		\subsection{Blackboard - SiSa - Email - Lessenrooster - WINAK - Printen - Cursusdienst}
			\begin{flushright}
				Robbe \& Joey
			\end{flushright}
			De allereerste sessie geven we uitleg over de voornaamste dingen die de studenten dan moeten weten. Dit houdt in:
			\begin{itemize}
				\item Uitleg over wat deze sessies inhouden en de planning.
				\item Een introductie tot Blackboard en SiSa.
				\item Het belang van de studenten e-mail en hoe je deze op andere apparaten kan instellen.
				\item Je lessenrooster met je persoonlijke agenda verbinden (bv.: op smartphone).
				\item Wie of wat is WINAK en wat kunnen ze voor jou betekenen (bv.: tuyaux).
				\item Uitleg over de andere services zoals de printkaart en de cursusdienst.
			\end{itemize}
			Aan het einde vermelden we zeker dat de volgende sessie meer praktijk-gericht is. Als de studenten hun laptop willen Dual Booten moeten ze hun laptop en een USB stick meenemen.
			
		\subsection{Dual Boot - Virtual Box (practicum)}
			\begin{flushright}
				Stijn \& Robbe
			\end{flushright}
			Op de universiteit gebruiken we Linux als referentieplatform. Alle projecten moeten hierop functioneren. Praktijkexamen gebeuren ook op Linux. Het kan dus alleen maar van pas komen als je op je eigen computer ook met Linux kan werken. In deze praktijksessie is het de bedoeling dat de ge\"interesseerden zelf hun laptop en een USB stick meebrengen. We begeleiden de studenten stap voor stap bij het installeren van Linux naast hun bestaande OS (Dual Boot). We leggen ook uit hoe Virtual Box werkt. Dit is een minder drastische manier om toch van thuis uit met Linux te kunnen werken.
			
		\subsection{Linux tools - VIM - Nano - IDE's}
			\begin{flushright}
				Tim \& Stijn
			\end{flushright}
			De vorige sessie hielpen we de studenten bij het installeren van Linux. In deze sessie focussen we ons op het installeren en toelichten van enkele programma's. We geven uitleg over de Linux commandline. Simpele commando's als \emph{cd, ls, rm, mkdir, touch, screen, chmod, man, cat, apt-get} komen zeker aan bod. Daarna volgt een beknopte introductie tot VIM en Nano. Dit zijn 2 teksteditors die volledig in de terminal werken. Als laatste lichten we toe wat IDE's zijn met als voorbeelden PyCharm en Eclipse.
						
		\subsection{Git - LaTeX}
			\begin{flushright}
				Joey \& Tim
			\end{flushright}
			Bij een programmeeropdracht waarvoor je moet samenwerken is git echt onmisbaar. Met dit programma kan je jouw code op een server opslaan. Een voordeel hiervan is dat je steeds naar vorige versies van je code kan zien. Mocht je computer crashen of je harddrive stuk gaan heb je nog steeds een backup van je project. Het voornaamste sterktepunt van git is dat je met meerdere mensen tegelijkertijd aan je project kan werken. Git zorgt er namelijk voor dat wijzigingen samengevoegd worden en dat iedereen met de meest recente versie van de code werkt. Wij leggen uit hoe je git gebruikt met bitbucket of github. \\
			De resterende tijd gebruiken we om het principe van LaTeX uit te leggen. De meeste mensen kennen enkel de \emph{What you see is what you get} (wysiwyg) editors. LaTeX daarentegen is een soort van declaratieve programmeertaal die enorm van pas komt bij het schrijven van verslagen.
		
		\subsection{Student Pack - Office - SSH/SFTP - Studento - Make}
			\begin{flushright}
				Stan \& Joey
			\end{flushright}
			Om af te sluiten vermelden we nog enkele van de services die de universiteit levert. Elke student heeft bijvoorbeeld recht op het github student pack. Een gratis exemplaar van Office wordt ook aangeboden zolang je studeert. De universiteit heeft voor haar studenten 24/7 een server draaien waar je gebruik van kan maken. We leggen uit hoe je hier gebruik van kan maken via \emph{ssh} en \emph{sftp (FileZilla)} \\
			Als er daarna nog tijd over is kunnen we nog wat uitleg geven over \emph{make}. Als voorbeeld gebruiken we dot files of een C++ project.
			
	\section{Voorbereidingen}
		Om dit project praktisch voor te bereiden moeten we nog enkele dingen regelen. Om te beginnen zouden we graag de sessies in het lessenrooster van de studenten krijgen. Een manier om de planning en inhoud van elke sessie door te geven zou ook handig zijn. Bijvoorbeeld een vak op Blackboard. Het vak computersystemen en -architectuur geeft ook les omtrent enkele features van Linux. Soms wordt ook een introductie gegeven over LaTeX en VIM. We willen niet meermaals dezelfde informatie geven, daarom kunnen we best contact opnemen met professor Vangheluwe.
		
\end{document}